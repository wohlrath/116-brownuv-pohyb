\section*{Diskuze}
Částice v substrátu se nepohybovaly jen v pozorovací rovině, takže bylo nutno na ně během jejich pohybu zaostřovat.
Bohužel mikroskop se viklal a při zaostřování se obraz posouval.
Tato závada byla objevena až po čtvrtém měření, takže přiložené měření č. 1 a 3 jsou touto chybou ještě postiženy, což mělo pravděpodobně za následek jejich nepřílišnou shodu s \eqref{eq::pomerdrah1234}.
Poté jsme zvětšili hloubku ostrosti mikroskopu, aby se během pozorování pohybu částice nemuselo zaostřovat.

Při měřeních, které nejsou přiloženy, bylo zaznamenáno příliš málo poloh (méně než 50), vyskytl se preferovaný směr tečení, nebo jsme pozorovali shluk částic (střední kvadratická dráha byla výrazně menší než u měření č. 1, 3, 8).

Tabelovaná hodnota Avogadrovy konstanty \cite{avogadro} je $\num{6.022}\times 10^{23}\,\si{\per\mole}$.
Námi naměřená hodnota se od této hodnoty liší o \SI{31}{\percent}.
Shodu považujeme za dostatečnou vzhledem k velmi vysoké nepřesnosti naší metody.