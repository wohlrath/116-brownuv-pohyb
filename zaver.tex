\section*{Závěr}
Pozorovali jsme Brownův pohyb částic latexu ve vodě.
Ověřili jsme Einsteinův vztah pro střední kvadratické posunutí v čase, při měření č. 8 jsme naměřili za časy $t$, $2t$, $3t$ a $4t$ poměr středních kvadratických posunutí $1:\num[separate-uncertainty=false]{2.06(27)}:\num[separate-uncertainty=false]{3.09(40)}:\num[separate-uncertainty=false]{4.24(54)}$.
V závorkách jsou uvedeny standardní odchylky průměru v posledním uvedeném řádu.
Při ostatních měřeních nebyly výsledky přesvědčivé.

Střední kvadratické posunutí za čas $t=\SI{4.8}{\s}$ při měření č. 8 bylo \SI{18(2)}{\micro\metre\squared}.

Aktivita Brownova pohybu částic latexu ve vodě za pokojové teploty je \SI{1.87(21)}{\micro\metre\squared\per\s}.

Z aktivity Brownova pohybu jsme určili Avogadrovu konstantu $N_A = (7,9 \pm 1,3) \times 10^{23} \,\si{\per\mole}$. 
Tabelovaná hodnota \cite{avogadro} je $\num{6.022}\times 10^{23}\,\si{\per\mole}$. Naše hodnota se s ní v rámci jedné směrodatné odchylky sice neshoduje, nicméně řád jsme určili přesně.